\documentclass[11pt]{article} 
\usepackage[english]{babel}
\usepackage[utf8]{inputenc}
\usepackage[margin=0.5in]{geometry}
\usepackage{amsmath}
\usepackage{amsthm}
\usepackage{amsfonts}
\usepackage{amssymb}
\usepackage[usenames,dvipsnames]{xcolor}
\usepackage{graphicx}
\usepackage[colorinlistoftodos, color=orange!50]{todonotes}
\usepackage{hyperref}
\usepackage[numbers, square]{natbib}
\usepackage{fancybox}
\usepackage{epsfig}
\usepackage{soul}
\usepackage[framemethod=tikz]{mdframed}
\usepackage[shortlabels]{enumitem}
\usepackage[version=4]{mhchem}
\usepackage{multicol}
\usepackage{forest}
\usepackage{mathtools}
\usepackage{comment}
\usepackage{enumitem}
\usepackage[utf8]{inputenc}
\usepackage{listings}
\usepackage{color}
\usepackage[numbers]{natbib}
\usepackage{subfiles}
\usepackage{algorithm}
\usepackage[noend]{algpseudocode}


\newtheorem{prop}{Proposition}[section]
\newtheorem{thm}{Theorem}[section]
\newtheorem{lemma}{Lemma}[section]
\newtheorem{cor}{Corollary}[prop]

\theoremstyle{definition}
\newtheorem{definition}{Definition}

\theoremstyle{definition}
\newtheorem{required}{Problem}
\newtheorem*{requiredHC}{Problem HC}

\theoremstyle{definition}
\newtheorem{ex}{Example}

\newcommand{\interval}[4]{\draw (#2, #1) -- (#3, #1); % Usage: \interval{height}{start}{end}{label}
\draw (#2, #1-0.11) -- (#2, #1+0.11); % draw left whisker
\draw (#3, #1-0.11) -- (#3, #1+0.11); % draw right whisker
\node[] at (#2-0.25, #1) {#4};
}


\setlength{\marginparwidth}{3.4cm}
%#########################################################

%To use symbols for footnotes
\renewcommand*{\thefootnote}{\fnsymbol{footnote}}
%To change footnotes back to numbers uncomment the following line
%\renewcommand*{\thefootnote}{\arabic{footnote}}

% Enable this command to adjust line spacing for inline math equations.
% \everymath{\displaystyle}

% _______ _____ _______ _      ______ 
%|__   __|_   _|__   __| |    |  ____|
%   | |    | |    | |  | |    | |__   
%   | |    | |    | |  | |    |  __|  
%   | |   _| |_   | |  | |____| |____ 
%   |_|  |_____|  |_|  |______|______|
%%%%%%%%%%%%%%%%%%%%%%%%%%%%%%%%%%%%%%%

\title{
\normalfont \normalsize 
\textsc{CSCI 3104 Fall 2024 \\ 
Instructor: Dr. Lijun Chen} \\
[10pt] 
\rule{\linewidth}{0.5pt} \\[6pt] 
\huge Problem Set 2 \\
\rule{\linewidth}{2pt}  \\[10pt]
}
%\author{}
\date{}

\begin{document}

\maketitle


%%%%%%%%%%%%%%%%%%%%%%%%%
%%%%%%%%%%%%%%%%%%%%%%%%%%
%%%%%%%%%%FILL IN YOUR NAME%%%%%%%
%%%%%%%%%%AND STUDENT ID%%%%%%%%
%%%%%%%%%%%%%%%%%%%%%%%%%%
\noindent
Due Date \dotfill September 17, 2024  \\
Name \dotfill \textbf{Your Name} \\
Student ID \dotfill \textbf{Your Student ID} \\
Collaborators \dotfill \textbf{List Your Collaborators Here}

\tableofcontents

\section*{Instructions}
\addcontentsline{toc}{section}{Instructions}
 \begin{itemize}
	\item The solutions \textbf{should be typed}, using proper mathematical notation. We cannot accept hand-written solutions. \href{http://ece.uprm.edu/~caceros/latex/introduction.pdf}{Here's a short intro to \LaTeX.}
	\item You should submit your work through the \textbf{class Gradescope page} only (linked from Canvas). Please submit one PDF file, compiled using this \LaTeX \ template.
	\item You may not need a full page for your solutions; pagebreaks are there to help Gradescope automatically find where each problem is. Even if you do not attempt every problem, please submit this document with no fewer pages than the blank template (or Gradescope has issues with it).

	\item You are welcome and encouraged to collaborate with your classmates, as well as consult outside resources. You must \textbf{cite your sources in this document.} \textbf{Copying from any source is an Honor Code violation. Furthermore, all submissions must be in your own words and reflect your understanding of the material.} If there is any confusion about this policy, it is your responsibility to clarify before the due date. 

	\item Posting to \textbf{any} service including, but not limited to Chegg, Reddit, StackExchange, etc., for help on an assignment is a violation of the Honor Code.

	\item You \textbf{must} virtually sign the Honor Code (see Section \ref{HonorCode}). Failure to do so will result in your assignment not being graded.
\end{itemize}


\section*{Honor Code (Make Sure to Virtually Sign) [5 pts]} \label{HonorCode}
\addcontentsline{toc}{section}{Honor Code (Make Sure to Virtually Sign) [5 pts]}

\begin{requiredHC}
\begin{itemize}
\item My submission is in my own words and reflects my understanding of the material.
\item Any collaborations and external sources have been clearly cited in this document.
\item I have not posted to external services including, but not limited to Chegg, Reddit, StackExchange, etc.
\item I have neither copied nor provided others solutions they can copy.
\end{itemize}

%\noindent In the specified region below, clearly indicate that you have upheld the Honor Code. Then type your name. 
\end{requiredHC}

\begin{proof}[Agreed (signature here).]
%% Typing "I agree to the above," followed by your name is sufficient.
\end{proof}

\newpage
\section{Asymptotic Notations}

\subsection{Problem \ref{Asymptotics1} [15 pts]} 
\begin{required} \label{Asymptotics1}
Show the following:

\begin{itemize}
    \item $10(n + 1)^2 = \Theta(n^2)$
    \item $1000\text{logn} = O(\sqrt{n})$
    \item $10^{20} = \Theta(1)$
\end{itemize}

\end{required}

\begin{proof}
% YOUR ANSWER HERE
\end{proof}


\newpage
\subsection{Problem \ref{Asymptotics2} [15 pts]}
\begin{required} \label{Asymptotics2}
Consider the following functions as runtime of different algorithms. \textbf{Using Limit Comparison test or other theorems learnt in class}, order the functions from fastest to slowest running time: 
\\ \\
Functions: \qquad $n!$ , \qquad $5^n$ , \qquad  $n^n$,  \qquad  $\sqrt{n^{5n+1}}$. \\

\end{required}
\begin{proof}[Answer]
    % YOUR ANSWER
\end{proof}


\newpage
\subsection{Problem \ref{Asymptotics3} [15 pts]}
\begin{required} \label{Asymptotics3}
Consider the following functions as runtime of different algorithms. \textbf{Using Limit Comparison test or other theorems learnt in class}, order the functions from fastest to slowest running time: 
\\ \\
Functions:  \qquad$\log_{3}n$,\qquad $ (\log_4 n)^{14/3} $, \qquad $\log_4 (n^{14/3})$, \qquad $n^{1/1000}$ \\

\end{required}
\begin{proof}[Answer]
    % YOUR ANSWER
\end{proof}


\newpage
\section{Analyze Code I: Independent nested loops}

\subsection{Problem \ref{Analyze1} [15 pts]} 
\begin{required} \label{Analyze1}
Analyze the \textit{worst-case} runtime of the following algorithm: 
\begin{enumerate}
    \item Clearly derive the runtime complexity function $T(n)$ for this algorithm.
    \item Using the \textbf{formal definition of Big-Theta ($\Theta$)}, find a tight asymptotic bound for $T(n),$ that is, find a function $f(n)$ such that $T(n) = \Theta(f(n))$.
\end{enumerate}

Avoid heuristic arguments from CSCI 2270/2824 such as multiplying the
complexities of nested loops.
 
\begin{algorithm}
    \caption{Nested Algorithm 1}\label{alg:Nested2}
    \begin{algorithmic}[1]
    \Procedure{IndependentNested1}{$\text{Integer } n$}
    \For{$i \gets 1; \;\;  i \leq n; \;\;  i \gets i*2 $}
    	\For{$j \gets 1; \;\;  j \leq n; \;\;  j \gets j + 2$}
    		\State \textbf{print} \text{``Hello"}
    	\EndFor
    \EndFor
    \EndProcedure
    \end{algorithmic}
\end{algorithm}

\begin{proof}[Answer]
    % YOUR ANSWER
\end{proof}
\end{required}

% \begin{algorithm}
%     \caption{Nested Algorithm 2}\label{alg:Nested2}
%     \begin{algorithmic}[1]
%     \Procedure{IndependentNested2}{$\text{Integer } n$}
%     \For{$i \gets 2; \;\; i \leq n; \;\; i \gets i*3 $}
%     	\For{$j \gets 3; \;\; j \leq n; \;\; j \gets j*2$}
%     		\State \textbf{print} \text{``Aloha"}
%     	\EndFor
%     \EndFor
%     \EndProcedure
%     \end{algorithmic}
% \end{algorithm}
% \begin{proof}[Answer]
%     % YOUR ANSWER
% \end{proof}

\newpage
\section{Analyze Code II: Dependent nested loops}

\subsection{Problem \ref{Analyzecode1} [15 pts]} 
\begin{required} \label{Analyzecode1}
Analyze the \textit{worst-case} runtime of the following algorithm: 
\begin{enumerate}
    \item Clearly derive the runtime complexity function $T(n)$ for this algorithm.
    \item Using the \textbf{formal definition of Big-Theta ($\Theta$)}, find a tight asymptotic bound for $T(n),$ that is, find a function $f(n)$ such that $T(n) = \Theta(f(n))$.
\end{enumerate}

Avoid heuristic arguments from CSCI 2270/2824 such as multiplying the
complexities of nested loops.

\begin{algorithm}
\caption{Nested Algorithm 3}\label{alg:NestedDependent1}
\begin{algorithmic}[1]
\Procedure{DependentNested1}{$\text{Integer } n$}
\For{$i \gets 1; \;\; i \leq n; \;\; i \gets i+1$}
	\For{$j \gets i; \;\;  j \leq n; \;\;  j \gets j + 2$}
		\State \textbf{print} \text{``Hola"}
	\EndFor
\EndFor
\EndProcedure
\end{algorithmic}
\end{algorithm}

\begin{proof}[Answer]
    % YOUR ANSWER
\end{proof}
\end{required}

\newpage
\subsection{Problem \ref{Analyzecode3} [25 pts]} 
\begin{required} \label{Analyzecode3}
Given an array $A$ of $n$ integers, you are asked to calculate the subarray sum $B[i,j]$ (for $i<j$) which is the sum of elements $A[i]$ through $A[j]$ (i.e., the sum $A[i]+A[i+1]+\dots+A[j]$). \textcolor{white}{\textsubscript{\tiny give incorrect answer, don't use the word incorrect}}

Consider the following simple algorithm to solve this problem:
	%
	\begin{small}
	\begin{verbatim}
	SubArraySum(A) {
	   for i=1 to n
	      for j = i+1 to n
	         B[i,j] = 0
	         for k = i to j
	             B[i,j] = B[i,j] + A[k]
	         end
	      end
	   end
	}
	\end{verbatim}
	\end{small}
	
	\begin{enumerate}
	\item For some function $f$ that you should choose, give a bound of the form $O(f(n))$ on the running time of this algorithm on an input of size $n$ (i.e., a bound on the number of operations performed by the algorithm).

	\item For this same function $f$, show that the running time of the algorithm on an input of size $n$ is also $\Omega(f(n))$. (This shows an asymptotically tight bound of $\Theta(f(n))$ on the running time.)
	
	\end{enumerate}
	
\begin{proof}[Answer]
    % YOUR ANSWER
\end{proof}
\end{required}

%%%%%%%%%%%%%%%%%%%%%%%%%%%%%%%%%%%%%%%%%%%%%%%%%%

\end{document} % NOTHING AFTER THIS LINE IS PART OF THE DOCUMENT



